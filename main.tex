%!TEX root = main-beamer.tex
%###############################################################################
\begin{document}
%###############################################################################

\maketitle           % makes title for article only

% Create a title slide
\begin{frame}[title]
  \titlepage
  % We cover slide number with a white rectangle
  \begin{tikzpicture}[remember picture,overlay] 
    \node [xshift=-9mm,yshift=6.5mm] at (current page.south east)
        {
          \begin{tikzpicture}
            \draw [white, fill=white] (0,0) rectangle (1.6, 0.6);
          \end{tikzpicture}
        };
  \end{tikzpicture}
  \setcounter{framenumber}{0}
\end{frame}

% Structuring a talk is a difficult task and the following structure
% may not be suitable. Here are some rules that apply for this
% solution: 

% - Exactly two or three sections (other than the summary).
% - At *most* three subsections per section.
% - Talk about 30s to 2min per frame. So there should be between about
%   15 and 30 frames, all told.

% - A conference audience is likely to know very little of what you
%   are going to talk about. So *simplify*!
% - In a 20min talk, getting the main ideas across is hard
%   enough. Leave out details, even if it means being less precise than
%   you think necessary.
% - If you omit details that are vital to the proof/implementation,
%   just say so once. Everybody will be happy with that.

\section*{Introduction}
%======================
\begin{frame}{Introduction}
  
\end{frame}
% #############################################################################


\section{Motivation}
%===================
\begin{frame}{Motivation – why we do this}

\end{frame}
% #############################################################################


\section{Materials and Methods}
%==============================
\begin{frame}{Materials}
             {And methods as well}

\end{frame}
  \subsection{Concepts}
  %--------------------

  \subsection{Experimental techniques}
  %--------------------
% #############################################################################


\section{Results and Contribution}
%=================================
\begin{frame}{Results and Contribution}

\end{frame}
% #############################################################################


\section*{Summary}
%=================
\begin{frame}{Summary}
  % Keep the summary *very short*.
  \begin{itemize}
  \item[\cmark] We again mention our cool results
  \item We mention something important
  \end{itemize}
  
  % The following outlook is optional.
  \vskip0pt plus.5fill
  \begin{itemize}
  \item
    Outlook
    \begin{itemize}
    \item[\xmark]
      Something you haven't solved.
    \item
      Something else you haven't solved.
    \end{itemize}
  \item[\ding{43}] We acknowledge some people for their help
  \end{itemize}
\end{frame}
% #############################################################################


% ########### Thanks ##########################################################
  \begin{frame}
    \begin{center}
      \begin{minipage}{0.7\textwidth}
        \begin{block}{}
          \centering
          \vspace{3mm}
          \Large \alert{Thank you for your attention!}
          \vspace{3mm}
        \end{block}
      \end{minipage}
    \end{center}
  \end{frame}

      



















%###############################################################################
% All of the following is optional and typically not needed. 
\appendix
\section<presentation>*{\appendixname}
\subsection<presentation>*{For Further Reading}

\begin{frame}[allowframebreaks]
  \frametitle<presentation>{For Further Reading}
    
  \begin{thebibliography}{10}
    
  \beamertemplatebookbibitems
  % Start with overview books.

  \bibitem{Author1990}
    A.~Author.
    \newblock {\em Handbook of Everything}.
    \newblock Some Press, 1990.
 
    
  \beamertemplatearticlebibitems
  % Followed by interesting articles. Keep the list short. 

  \bibitem{Someone2000}
    S.~Someone.
    \newblock On this and that.
    \newblock {\em Journal of This and That}, 2(1):50--100,
    2000.
  \end{thebibliography}
\end{frame}

\begin{frame}{Beamer tools -- use them}
  \begin{itemize}
  \item block
  \item theorem
  \item<2> example
  \item[$\checkmark$] proof
  \item description
  \item \alert{important stuff}
  \item use columns
  \item \textbackslash{}framezoom
  \end{itemize}
\end{frame}


\begin{frame}{Enumerate, Itemize}
  There are three important points:
  \begin{enumerate}
    \item<1-> A first one,
    \item<2-> a second one with a bunch of subpoints,
  \begin{itemize}
    \item first subpoint. (Only shown from second slide on!).
    \item<3-> second subpoint added on third slide.
    \item<4-> third subpoint added on fourth slide.
  \end{itemize}
    \item<5-> and a third one.
  \end{enumerate}
\end{frame}


\begin{frame}{Structure}
  We have some text and then use \structure{\textbackslash{}structure} in it.
\end{frame}

\begin{frame}{Block}
  block of text that has a heading
  \setbeamertemplate{blocks}[default]
%  \begin{block}<⟨action specification⟩>{⟨block title⟩}<⟨action specification⟩>
  \begin{block}{block title}
  environment contents
  \end{block}
  \setbeamertemplate{blocks}[rounded][shadow=true]
  \begin{block}{block title, rounded block with shadow}
  environment contents
  \end{block}
  \begin{alertblock}{alertblock}
  environment contents
  \end{alertblock}
  \begin{exampleblock}{exampleblock}
  environment contents
  \end{exampleblock}
\end{frame}


\begin{frame}{Theorem, definition, proof}
  \begin{theorem}<1->[Theorem -- Additional text]
  There exists an infinite set.
  \end{theorem}
  \begin{definition}<1->[Definition -- Additional text]
  There exists an infinite set.
  \end{definition}
  \begin{proof}<2->[Proof -- Additional text]
  This follows from the axiom of infinity.
  \end{proof}
  \begin{example}<3->[Natural Numbers]
  The set of natural numbers is infinite.
  \end{example}
\end{frame}




\begin{frame}{Framed and boxed text}
  Text without box
  \begin{beamercolorbox}{beamer color box}
    Text in beamercolorbox
  \end{beamercolorbox}
  
  \setbeamercolor{postit}{fg=black,bg=yellow}
  % Options can be: wd, dp, ht, left, right, center, leftskip, rightskip,
  %                 sep, colsep, colsep*, shadow, rounded, ignorebg, vmode
  \begin{beamercolorbox}[sep=1em,wd=5cm]{postit}
    Place me somewhere!
  \end{beamercolorbox}
  
  \shadowbox{shadowbox}, \doublebox{doublebox}, \ovalbox{ovalbox}, and \Ovalbox{Ovalbox}
\end{frame}


\begin{frame}{Columns}
% Column options:
  % b  will cause the bottom lines of the columns to be vertically aligned
  % c  will cause the columns to be centered vertically relative to each other.
  %       Default, unless the global option t is used.
  % onlytextwidth  is the same as totalwidth=\textwidth.
  % t  will cause the first lines of the columns to be aligned. Default if global option t is used.
  % T  is similar to the t option, but T aligns the tops of the first lines while t
  %       aligns the so-called baselines of the first lines.
  % totalwidth=⟨width⟩  will cause the columns to occupy not the whole page width,
  %       but only ⟨width⟩, all told.
  \begin{columns}[t]
    \begin{column}{5cm}
    Two\\lines.
    \end{column}
    \begin{column}{5cm}
    One line (but aligned).
    \end{column}
  \end{columns}
\end{frame}


\begin{frame}{Columns}
% \column  command
  % Starts a single column. The parameters and options are the same as for the column environment. The
  % column automatically ends with the next occurrence of \column or of a column environment or of the end
  % of the current columns environment.
  \begin{columns}
    \column[t]{5cm}
      Two\\lines.
    \column[t]{5cm}
      One line (but aligned).
  \end{columns}
\end{frame}


\begin{frame}{Abstract}
  \begin{abstract}
    This is the abstract
  \end{abstract}
\end{frame}

\begin{frame}{Columns with figure}
% \column  command
  % Starts a single column. The parameters and options are the same as for the column environment. The
  % column automatically ends with the next occurrence of \column or of a column environment or of the end
  % of the current columns environment.
  \begin{columns}
    \column[c]{0.45\textwidth}
      \begin{figure}
        \includegraphics[width=4cm]{example.png}
        \caption[my caption]{An example image}
      \end{figure}
    \column[c]{0.45\textwidth}
      Look at this image. What do you see?
      \begin{itemize}
       \item Item 1
       \item Item 2
      \end{itemize}
  \end{columns}
\end{frame}


\begin{frame}{Verse and quotations}
  \begin{verse}
    This is inside of verse
  \end{verse}
  \begin{quotation}
    This is inside of quotation
  \end{quotation}
  \begin{quote}
    This is inside of quote
  \end{quote}
\end{frame}


\begin{frame}{Slide transitions}
  \begin{enumerate}
    \item<1,2> First slide transition
    \item<2> Second slide transition
    \item<3> Third slide transition
  \end{enumerate}
  \transblindshorizontal<2>
  \transblindsvertical<3>
% Other transitions are:
%\transboxin, \transboxout, \transdissolve, \transdissolve, \transglitter[direction=90], 
%\transsplitverticalin, \transsplitverticalout, \transsplithorizontalin, \transsplithorizontalin,
%\transsplithorizontalout, \transwipe, \transwipe, \transduration
\end{frame}


%###############################################################################
\end{document}
%###############################################################################
