%!TEX root = main-beamer.tex
%\documentclass[ignorenonframetext]{beamer} % Just to fool texstudio
% Presentation for NANOMETA-2015
% Author: Roman Kiselev, roman.kiselew@gmail.com

%-------------------------------------------------------------------------------
%                           Necessary packages
%-------------------------------------------------------------------------------
\usepackage{graphicx}          % graphics
\usepackage{hyperref}          % urls
\usepackage{pgf}               % for images
\usepackage{fancybox}          % for \shadowbox, \doublebox, \ovalbox, and \Ovalbox
\usepackage{pgfpages}
%\usepackage{amsmath}
%\usepackage{mhchem}            % chemical equations, \ce command
\usepackage{amsfonts}          % http://ctan.org/pkg/amsfonts
\usepackage{pifont}            % http://ctan.org/pkg/pifont
\newcommand{\cmark}{\ding{51}}%
\newcommand{\xmark}{\ding{55}}%
%\usepackage{booktabs, caption} % table styling
%\usepackage{xcolor}            % custom colors
%\definecolor{mypurple}{HTML}{8054cc}  % definition of a custom color

\graphicspath{ {./img/} }

%-------------------------------------------------------------------------------
%                           Second monitor?
%-------------------------------------------------------------------------------
%\setbeameroption{previous slide on second screen=left}

%-------------------------------------------------------------------------------
%                           Font specification
%-------------------------------------------------------------------------------
%\usepackage{fontspec}
%\setsansfont{Fontin}

%-------------------------------------------------------------------------------
%          Information about Title, Author, Affiliations
%-------------------------------------------------------------------------------
\title[Data transmission in LR-DLSPPW]
      {Data transmission in long-range dielectric-loaded surface plasmon 
      polariton waveguides}

\date[Nanometa-2015]{8-th of January, 2015}

% This is only inserted into the PDF information catalog. Can be left out
\keywords{photonics, nanooptics, fiber optic communication, telecom,
          bit-error rate, surface plasmon-polaritons, BER penalties}

% Titlegraphic appears in the middle of the titlepage
\pgfdeclareimage[height=0.5cm]{titlegraphic}{img/logo1}
%\titlegraphic{\pgfuseimage{titlegraphic}}

% Logo; use \insertlogo anywhere you want
\pgfdeclareimage[height=0.5cm]{university-logo}{img/logo2}
%\logo{\pgfuseimage{university-logo}}


%-------------------------------------------------------------------------------
%                                  Appearance
%-------------------------------------------------------------------------------
% Themes without navi bars: default, boxes, Bergen, Boadilla, Madrid, AnnArbor,
%                           CambridgeUS, Pittsburgh, Rochester
% Themes with tree-like bar: Antibes, JuanLesPins, Montpellier
% Themes with sidebar: Berkeley, PaloAlto, Goettingen, Marburg, Hannover
% Themes with mini frame navi: Berlin, Ilmenau, Dresden, Darmstadt, Frankfurt, Singapore, Szeged
% Themes with sections and subsections tables: Copenhagen, Luebeck, Malmoe, Warsaw, 
\usetheme{CambridgeUS}

% Themes: default, circles, rectangles, rounded, inmargin
\useinnertheme{default}

% Themes: default, infolines, miniframes, smoothbars, sidebar, split, shadow, tree, smoothtree
\useoutertheme{default}

% Special themes: default, structure, sidebartab
% Complete themes: albatross, beetle, crane, dove, fly, seagull, wolverine, beaver
% Inner themes: lily, orchid, rose
% Outer themes: whale, seahorse, dolphin, 
\usecolortheme{crane}

% % % % % Tweaks
%\setbeamercolor{frametitle}{fg=red}
\setbeamertemplate{caption}{\raggedright\insertcaption\par}

% Delete this, if you do not want the table of contents to pop up at
% the beginning of each subsection:
\AtBeginSection[]
{
  \begin{frame}<presentation>{Outline}
    \tableofcontents[currentsection,currentsubsection]
  \end{frame}
}

% Make hidden elements a bit visible
\setbeamercovered{transparent}

% If you wish to uncover everything in a step-wise fashion, uncomment
% the following command:
%\beamerdefaultoverlayspecification{<+->}

% Navigation symbols and frame numbers
\setbeamertemplate{navigation symbols}{
    \insertslidenavigationsymbol
    \insertframenavigationsymbol
    \insertsubsectionnavigationsymbol
    \insertsectionnavigationsymbol
    \insertdocnavigationsymbol
    \insertbackfindforwardnavigationsymbol}
\setbeamerfont{navigation symbols}{size=\normalsize}

% Width of blocks
\addtobeamertemplate{block begin}{%
  \setlength{\textwidth}{1.0\textwidth}%
}{}

\addtobeamertemplate{block alerted begin}{%
  \setlength{\textwidth}{1.0\textwidth}%
}{}

\addtobeamertemplate{block example begin}{%
  \setlength{\textwidth}{1.0\textwidth}%
}{}

%-------------------------------------------------------------------------------
%                         Bibliography
%-------------------------------------------------------------------------------
\usepackage[style=authoryear-icomp,
            uniquelist=false,
            uniquename = false,
            dashed=false,
            abbreviate=true,
            dateabbrev=true,
            maxcitenames=1,
            backend=bibtex]{biblatex}
\bibliography{links}

\DeclareBibliographyDriver{article}{%
	\usebibmacro{bibindex}%
	\usebibmacro{begentry}%
	\usebibmacro{author}%/translator+others}%
	\setunit{\labelnamepunct}\newblock
	%  \usebibmacro{title}%
	%  \newunit
	%  \printlist{language}%
	\newunit\newblock
	\usebibmacro{byauthor}%
	\newunit\newblock
	\usebibmacro{byeditor+others}%
	\newunit\newblock
	%\printfield{version}%
	\newunit\newblock
	% \usebibmacro{in:}%
	\usebibmacro{journal}%
	\newunit\newblock
	%\printfield{note}%
	\setunit{\bibpagespunct}%
	%\printfield{pages}
	%  \newunit\newblock
	%  \printfield{issn}%
	\newunit\newblock
	%\printfield{doi}%
	%  \newunit\newblock
	%  \usebibmacro{eprint:arxiv}
	%  \newunit\newblock
	%  \usebibmacro{url+urldate}%
	\newunit\newblock
	\printfield{addendum}%
	\newunit\newblock
	\usebibmacro{pageref}%
	\usebibmacro{finentry}}
\DefineBibliographyStrings{ngerman}{
	andothers = {\emph{et\addabbrvspace al\adddot}}            
}

\renewcommand{\cite}{\footfullcite}


%###############################################################################
\begin{document}
%###############################################################################

\maketitle           % makes title for article only
\frame{\titlepage}   % makes title for presentation only


% Structuring a talk is a difficult task and the following structure
% may not be suitable. Here are some rules that apply for this
% solution: 

% - Exactly two or three sections (other than the summary).
% - At *most* three subsections per section.
% - Talk about 30s to 2min per frame. So there should be between about
%   15 and 30 frames, all told.

% - A conference audience is likely to know very little of what you
%   are going to talk about. So *simplify*!
% - In a 20min talk, getting the main ideas across is hard
%   enough. Leave out details, even if it means being less precise than
%   you think necessary.
% - If you omit details that are vital to the proof/implementation,
%   just say so once. Everybody will be happy with that.


\section*{Introduction}
%----------------------
  We demonstrate data transmission of 10 Gbit/s on-off keying modulated
  1550 nm signal through a long-range dielectric-loaded surface
  plasmon polariton waveguide (LR-DLSSPw) structure with negligible signal
  degradation. In the experiment the bit error rate penalties do not exceed
  0.6 dB over the 15 nm wavelength range and received optical power
  between --7 and 3 dBm.
  \begin{frame}<presentation>{Abstract}
    We demonstrate 10\,Gbit/s transmission over 300\,$\mu$m-long nanooptic
    waveguide with almost negligible signal degradation.
  \end{frame}

\section{Motivation}
%-------------------
  In this section we briefly describe waveguides that we were using; we
  mention what are their possible application.
  \begin{frame}{Motivation -- Waveguide Structure}
    LR-DLSPPW -- long-range dielectric-loaded surface plasmon-polariton waveguide \cite{volkov_long-range_2011}
    \begin{columns}
      \column[c]{0.42\textwidth}
        \begin{figure}
          \includegraphics[width=43mm]{wg_profile.pdf}
        \end{figure}
      \column[c]{0.58\textwidth}
        \begin{itemize}
         \item Dielectric stripe on metallic one
         \item Tight sub-wavelength confinement
         \item mm-long propagation distance
        \end{itemize}
    \end{columns}
  \end{frame}
  
  On the next slide we present the problem~-- these waveguides have huge
  insertion losses. It is not clear if a modulated signal can be recovered
  after such waveguide.

  \begin{frame}{Motivation: properties of the LR-DLSPPW structure}
    \begin{columns}
      \column[t]{0.45\textwidth}
        \begin{block}{Key features:}
          \begin{itemize}
            \item Strong light confinement
            \item Low dissipation
            \item Loss compensation by stimulated emission of plasmonic modes
          \end{itemize}
        \end{block}
      \column[t]{0.45\textwidth}
        \begin{block}{Possible applications:}
          \begin{itemize}
            \item Compact telecom devices
            \item Integrated plasmonic circuits
            \item Optical interconnects
          \end{itemize}
        \end{block}
    \end{columns}
    \pause
    \begin{center}
      \begin{alertblock}{Transmission of modulated signal}
        Test LR-DLSSPW under the standard
        telecom environment
      \end{alertblock}      
    \end{center}
  \end{frame}


\section{Materials and Methods}
%------------------------------
  This section presents an overview of used materials and methods
  \begin{frame}{Methods}
    \begin{itemize}
      \item Transmit 10\,Gbit/s modulated signal at 1550\,nm
      \item Evaluate bit-error rate (BER) penalties
      \item Evaluate channel cross-talk
    \end{itemize}
    \pause
    \begin{exampleblock}{What are \emph{BER penalties}?}
      \emph{BER} -- probability of incorrect bit detection\\
      penalty -- \emph{BER} change between tested device and ideal attenuator
    \end{exampleblock}
  \end{frame}

  \begin{frame}{Measurement techniques}
   \begin{center}
    \includegraphics[width=\textwidth]{BERsetup.pdf}
   \end{center}
    Prototype of fiber optic telecommunication line:
    \begin{itemize}
      \item Transmitter
      \item Device under test
      \item Receiver
    \end{itemize}
  \end{frame}

  \begin{frame}{Data processing techniques}
    We process the data using:
    \begin{itemize}
      \item some cool statistics
      \item what we get from it
      \item show some plots
    \end{itemize}
    \label{slide:data_processing}
  \end{frame}


\begin{figure}
  \begin{center}
    \includeslide[height=5cm]{slide:data_processing}
  \end{center}
  \caption{The first slide (height 5cm). Note the partly covered second item.}
\end{figure}


\section{Results and Contribution}
%---------------------------------
  \begin{frame}<presentation>{Results and Contribution}
  To sum up, we have done this and that. These results are important and cool,
  they were published elsewhere [Author A, 2014].
  \label{slide:results}
  \end{frame}

%\article
\begin{figure}
  \begin{center}
    \includeslide[height=5cm]{slide:results}
  \end{center}
  \caption{The first slide (height 5cm). Note the partly covered second item.}
\end{figure}

\section*{Summary}
%-----------------
  \begin{frame}{Summary}
    % Keep the summary *very short*.
    \begin{itemize}
    \item[\cmark] We again mention our cool results
    \item We mention something important
    \end{itemize}
    
    % The following outlook is optional.
    \vskip0pt plus.5fill
    \begin{itemize}
    \item
      Outlook
      \begin{itemize}
      \item[\xmark]
        Something you haven't solved.
      \item
        Something else you haven't solved.
      \end{itemize}
    \item[\ding{43}] We acknowledge some people for their help
    \end{itemize}
  \end{frame}



%#############################################################################
% All of the following is optional and typically not needed. 
\appendix
\section<presentation>*{\appendixname}
\subsection<presentation>*{For Further Reading}

\begin{frame}[allowframebreaks]
  \frametitle<presentation>{For Further Reading}
    
  \begin{thebibliography}{10}
    
  \beamertemplatebookbibitems
  % Start with overview books.

  \bibitem{Author1990}
    A.~Author.
    \newblock {\em Handbook of Everything}.
    \newblock Some Press, 1990.
 
    
  \beamertemplatearticlebibitems
  % Followed by interesting articles. Keep the list short. 

  \bibitem{Someone2000}
    S.~Someone.
    \newblock On this and that.
    \newblock {\em Journal of This and That}, 2(1):50--100,
    2000.
  \end{thebibliography}
\end{frame}

\begin{frame}{Beamer tools -- use them}
  \begin{itemize}
  \item block
  \item theorem
  \item<2> example
  \item[$\checkmark$] proof
  \item description
  \item \alert{important stuff}
  \item use columns
  \item \textbackslash{}framezoom
  \end{itemize}
\end{frame}


\begin{frame}{Enumerate, Itemize}
  There are three important points:
  \begin{enumerate}
    \item<1-> A first one,
    \item<2-> a second one with a bunch of subpoints,
  \begin{itemize}
    \item first subpoint. (Only shown from second slide on!).
    \item<3-> second subpoint added on third slide.
    \item<4-> third subpoint added on fourth slide.
  \end{itemize}
    \item<5-> and a third one.
  \end{enumerate}
\end{frame}


\begin{frame}{Structure}
  We have some text and then use \structure{\textbackslash{}structure} it it.
\end{frame}

\begin{frame}{Block}
  block of text that has a heading
  \setbeamertemplate{blocks}[default]
%  \begin{block}<⟨action specification⟩>{⟨block title⟩}<⟨action specification⟩>
  \begin{block}{block title}
  environment contents
  \end{block}
  \setbeamertemplate{blocks}[rounded][shadow=true]
  \begin{block}{block title, rounded block with shadow}
  environment contents
  \end{block}
  \begin{alertblock}{alertblock}
  environment contents
  \end{alertblock}
  \begin{exampleblock}{exampleblock}
  environment contents
  \end{exampleblock}
\end{frame}


\begin{frame}{Theorem, definition, proof}
  \begin{theorem}<1->[Theorem -- Additional text]
  There exists an infinite set.
  \end{theorem}
  \begin{definition}<1->[Definition -- Additional text]
  There exists an infinite set.
  \end{definition}
  \begin{proof}<2->[Proof -- Additional text]
  This follows from the axiom of infinity.
  \end{proof}
  \begin{example}<3->[Natural Numbers]
  The set of natural numbers is infinite.
  \end{example}
\end{frame}




\begin{frame}{Framed and boxed text}
  Text without box
  \begin{beamercolorbox}{beamer color box}
    Text in beamercolorbox
  \end{beamercolorbox}
  
  \setbeamercolor{postit}{fg=black,bg=yellow}
  % Options can be: wd, dp, ht, left, right, center, leftskip, rightskip,
  %                 sep, colsep, colsep*, shadow, rounded, ignorebg, vmode
  \begin{beamercolorbox}[sep=1em,wd=5cm]{postit}
    Place me somewhere!
  \end{beamercolorbox}
  
  \shadowbox{shadowbox}, \doublebox{doublebox}, \ovalbox{ovalbox}, and \Ovalbox{Ovalbox}
\end{frame}


\begin{frame}{Columns}
% Column options:
  % b  will cause the bottom lines of the columns to be vertically aligned
  % c  will cause the columns to be centered vertically relative to each other.
  %       Default, unless the global option t is used.
  % onlytextwidth  is the same as totalwidth=\textwidth.
  % t  will cause the first lines of the columns to be aligned. Default if global option t is used.
  % T  is similar to the t option, but T aligns the tops of the first lines while t
  %       aligns the so-called baselines of the first lines.
  % totalwidth=⟨width⟩  will cause the columns to occupy not the whole page width,
  %       but only ⟨width⟩, all told.
  \begin{columns}[t]
    \begin{column}{5cm}
    Two\\lines.
    \end{column}
    \begin{column}{5cm}
    One line (but aligned).
    \end{column}
  \end{columns}
\end{frame}


\begin{frame}{Columns}
% \column  command
  % Starts a single column. The parameters and options are the same as for the column environment. The
  % column automatically ends with the next occurrence of \column or of a column environment or of the end
  % of the current columns environment.
  \begin{columns}
    \column[t]{5cm}
      Two\\lines.
    \column[t]{5cm}
      One line (but aligned).
  \end{columns}
\end{frame}


\begin{frame}{Abstract}
  \begin{abstract}
    This is the abstract
  \end{abstract}
\end{frame}

\begin{frame}{Columns with figure}
% \column  command
  % Starts a single column. The parameters and options are the same as for the column environment. The
  % column automatically ends with the next occurrence of \column or of a column environment or of the end
  % of the current columns environment.
  \begin{columns}
    \column[c]{0.45\textwidth}
      \begin{figure}
        \includegraphics[width=\textwidth]{example.png}
        \caption[my caption]{An example image}
      \end{figure}
    \column[c]{0.45\textwidth}
      Look at this image. What do you see?
      \begin{itemize}
       \item Item 1
       \item Item 2
      \end{itemize}
  \end{columns}
\end{frame}


\begin{frame}{Verse and quotations}
  \begin{verse}
    This is inside of verse
  \end{verse}
  \begin{quotation}
    This is inside of quotation
  \end{quotation}
  \begin{quote}
    This is inside of quote
  \end{quote}
\end{frame}


\begin{frame}{Slide transitions}
  \begin{enumerate}
    \item<1,2> First
    \item<2> Second
    \item<3> Third
  \end{enumerate}
  \transblindshorizontal<2>
  \transblindsvertical<3>
% Other transitions are:
%\transboxin, \transboxout, \transdissolve, \transdissolve, \transglitter[direction=90], 
%\transsplitverticalin, \transsplitverticalout, \transsplithorizontalin, \transsplithorizontalin,
%\transsplithorizontalout, \transwipe, \transwipe, \transduration
\end{frame}


%###############################################################################
\end{document}
%###############################################################################