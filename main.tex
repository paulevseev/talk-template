% Template for BEAMER presentation
%
% Author: Roman Kiselev, roman.kiselew@gmail.com
%
% Requirements:
%  * xelatex
%  * biber
%  * Fontin font (http://www.exljbris.com/fontinsans.html)


%-------------------------------------------------------------------------------
%                           Necessary packages
%-------------------------------------------------------------------------------
\usepackage{graphicx}          % graphics
\usepackage{hyperref}          % urls
\usepackage{pgf}               % for images
\usepackage{fancybox}          % for \shadowbox, \doublebox, \ovalbox, and \Ovalbox
%\usepackage{booktabs, caption} % table styling
%\usepackage{amsmath}
%\usepackage{mhchem}            % chemical equations, \ce command
%\usepackage{xcolor}            % custom colors
%\definecolor{mypurple}{HTML}{8054cc}  % definition of a custom color

\graphicspath{ {./img/} }

%-------------------------------------------------------------------------------
%                           Font specification
%-------------------------------------------------------------------------------
%\usepackage{fontspec}
%\setsansfont{Fontin}
%\setbeamerfont{frametitle}{size=\LARGE,series=\bfseries}


%-------------------------------------------------------------------------------
%          Information about Title, Author, Affiliations
%-------------------------------------------------------------------------------
\title[Short title of the talk]
      {Full and really long title of the talk}

\subtitle{PhD progress report or conference talk, I don't know yet}

\date[CONFname]{YYYY-MM-DD}

% This is only inserted into the PDF information catalog. Can be left out
\keywords{microscopy, Raman, spectroscopy, imaging, RACS, tumour cells}

% Use template below if you are the only author
%\author[R. Kiselev]{Roman Kiselev}
%\institute[IPHT Jena]{
%  Optical cell diagnostics\\
%  Leibniz Institute of Photonic Technology
%  }



% Titlegraphic appears in the middle of the titlepage
\pgfdeclareimage[height=0.5cm]{titlegraphic}{img/logo1}
\titlegraphic{\pgfuseimage{titlegraphic}}

% Logo; use \insertlogo anywhere you want
\pgfdeclareimage[height=0.5cm]{university-logo}{img/logo2}
\logo{\pgfuseimage{university-logo}}


%-------------------------------------------------------------------------------
%                                  Appearance
%-------------------------------------------------------------------------------
\usetheme{Warsaw}
\usecolortheme{crane}


%###############################################################################
\begin{document}
%###############################################################################

\maketitle           % makes title for article only
\frame{\titlepage}   % makes title for presentation only


% Structuring a talk is a difficult task and the following structure
% may not be suitable. Here are some rules that apply for this
% solution: 

% - Exactly two or three sections (other than the summary).
% - At *most* three subsections per section.
% - Talk about 30s to 2min per frame. So there should be between about
%   15 and 30 frames, all told.

% - A conference audience is likely to know very little of what you
%   are going to talk about. So *simplify*!
% - In a 20min talk, getting the main ideas across is hard
%   enough. Leave out details, even if it means being less precise than
%   you think necessary.
% - If you omit details that are vital to the proof/implementation,
%   just say so once. Everybody will be happy with that.


\section*{Introduction}
%---------------------
  What topic will cover this talk and what results will be presented. This info is
  necessary to attract attention of listeners. Place some nice eye-capturing figure here.

\section{Motivation}
%-------------------
  Briefly mention some basic technical terms and methods relative to your field / your study.

\subsection{The Basic Problem That We Studied}
%---------------------------------------------
  Here we present the problem that we faced in our study.

  \begin{frame}{The Problem}
    You can create overlays\dots
    \begin{itemize}
    \item using the \texttt{pause} command:
      \begin{itemize}
      \item
        First item.
        \pause
      \item    
        Second item.
      \end{itemize}
    \item
      using overlay specifications:
      \begin{itemize}
      \item<3->
        First item.
      \item<4->
        Second item.
      \end{itemize}
    \item
      using the general \texttt{uncover} command:
      \begin{itemize}
        \uncover<5->{\item
          First item.}
        \uncover<6->{\item
          Second item.}
      \end{itemize}
    \end{itemize}
  \end{frame}


\section{Materials and Methods}
%------------------------------
  This section presents an overview of used materials and methods
  \begin{frame}<presentation>{Methods}
    This slide presents materials and methods\ldots
    \begin{itemize}
      \item Place here a lot of figures and
      \item \textbackslash{}items
    \end{itemize}
  \end{frame}

  \begin{frame}{Measurement techniques}
    We used the following experimental setup
    \begin{itemize}
      \item how it works
      \item what it does
    \end{itemize}
  \end{frame}

  \begin{frame}{Data processing techniques}
    We process the data using:
    \begin{itemize}
      \item some cool statistics
      \item what we get from it
      \item show some plots
    \end{itemize}
  \end{frame}


\section{Results and Contribution}
%---------------------------------
  \begin{frame}{Results and Contribution}
  To sum up, we have done this and that. These results are important and cool,
  they were published elsewhere [Author A, 2014].
  \end{frame}


\section*{Summary}
%-----------------
  \begin{frame}{Summary}
    \begin{itemize}
      \item We again mention our cool results
      \item We acknowledge some people for their help
    \end{itemize}
  \end{frame}



%#############################################################################
% All of the following is optional and typically not needed. 
\appendix
\section<presentation>*{\appendixname}
\subsection<presentation>*{For Further Reading}

\begin{frame}[allowframebreaks]
  \frametitle<presentation>{For Further Reading}
    
  \begin{thebibliography}{10}
    
  \beamertemplatebookbibitems
  % Start with overview books.

  \bibitem{Author1990}
    A.~Author.
    \newblock {\em Handbook of Everything}.
    \newblock Some Press, 1990.
 
    
  \beamertemplatearticlebibitems
  % Followed by interesting articles. Keep the list short. 

  \bibitem{Someone2000}
    S.~Someone.
    \newblock On this and that.
    \newblock {\em Journal of This and That}, 2(1):50--100,
    2000.
  \end{thebibliography}
\end{frame}

\begin{frame}{Beamer tools -- use them}
  \begin{itemize}
  \item block
  \item theorem
  \item<2> example
  \item[$\checkmark$] proof
  \item description
  \item \alert{important stuff}
  \item use columns
  \item \textbackslash{}framezoom
  \end{itemize}
\end{frame}


\begin{frame}{Enumerate, Itemize}
  There are three important points:
  \begin{enumerate}
    \item<1-> A first one,
    \item<2-> a second one with a bunch of subpoints,
  \begin{itemize}
    \item first subpoint. (Only shown from second slide on!).
    \item<3-> second subpoint added on third slide.
    \item<4-> third subpoint added on fourth slide.
  \end{itemize}
    \item<5-> and a third one.
  \end{enumerate}
\end{frame}


\begin{frame}{Structure}
  We have some text and then use \structure{\textbackslash{}structure} it it.
\end{frame}

\begin{frame}{Block}
  block of text that has a heading
  \setbeamertemplate{blocks}[default]
%  \begin{block}<⟨action specification⟩>{⟨block title⟩}<⟨action specification⟩>
  \begin{block}{block title}
  environment contents
  \end{block}
  \setbeamertemplate{blocks}[rounded][shadow=true]
  \begin{block}{block title, rounded block with shadow}
  environment contents
  \end{block}
  \begin{alertblock}{alertblock}
  environment contents
  \end{alertblock}
  \begin{exampleblock}{exampleblock}
  environment contents
  \end{exampleblock}
\end{frame}


\begin{frame}{Theorem, definition, proof}
  \begin{theorem}<1->[Theorem -- Additional text]
  There exists an infinite set.
  \end{theorem}
  \begin{definition}<1->[Definition -- Additional text]
  There exists an infinite set.
  \end{definition}
  \begin{proof}<2->[Proof -- Additional text]
  This follows from the axiom of infinity.
  \end{proof}
  \begin{example}<3->[Natural Numbers]
  The set of natural numbers is infinite.
  \end{example}
\end{frame}




\begin{frame}{Framed and boxed text}
  Text without box
  \begin{beamercolorbox}{beamer color box}
    Text in beamercolorbox
  \end{beamercolorbox}
  
  \setbeamercolor{postit}{fg=black,bg=yellow}
  % Options can be: wd, dp, ht, left, right, center, leftskip, rightskip,
  %                 sep, colsep, colsep*, shadow, rounded, ignorebg, vmode
  \begin{beamercolorbox}[sep=1em,wd=5cm]{postit}
    Place me somewhere!
  \end{beamercolorbox}
  
  \shadowbox{shadowbox}, \doublebox{doublebox}, \ovalbox{ovalbox}, and \Ovalbox{Ovalbox}
\end{frame}


%###############################################################################
\end{document}
%###############################################################################
