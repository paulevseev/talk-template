%!TEX root = main-beamer.tex
% Author: Roman Kiselev, roman.kiselew@gmail.com
%
% Requirements:
%  * xelatex
%  * biber

%-------------------------------------------------------------------------------
%                           Necessary packages
%-------------------------------------------------------------------------------
\usepackage{graphicx}          % graphics
%\usepackage{transparent}       % to make images semi-transparent
\usepackage{hyperref}          % urls
\usepackage{pgf}               % for images
\usepackage{pgfpages}
\usepackage{fancybox}          % for \shadowbox, \doublebox, \ovalbox, and \Ovalbox
\usepackage{tikz}              % TikZ images
%\usepackage{soul}              % striked out text, \st{} command
%\usepackage{amsmath}
%\usepackage{mhchem}            % chemical equations, \ce{} command
\usepackage{amsfonts}          % http://ctan.org/pkg/amsfonts
%\usepackage{booktabs, caption} % table styling
\usepackage{xcolor}            % custom colors
\definecolor{wheat}{HTML}{F5DEB3}  % definition of a custom color

\usepackage{pifont}            % http://ctan.org/pkg/pifont
\newcommand{\cmark}{\ding{51}}%
\newcommand{\xmark}{\ding{55}}%

\graphicspath{ {./img/} }

%-------------------------------------------------------------------------------
%                           Second monitor?
%-------------------------------------------------------------------------------
%\setbeameroption{previous slide on second screen=left}

%-------------------------------------------------------------------------------
%                           Font specification
%-------------------------------------------------------------------------------
% Require XeTeX if uncommented
\usepackage{fontspec}
\setsansfont{Linux Biolinum O}  % Fontin, Linux Biolinum O

%-------------------------------------------------------------------------------
%          Information about Title, Author, Affiliations
%-------------------------------------------------------------------------------
\title[Short title of the talk]
      {Full and really long title of the talk}

\subtitle{PhD progress report or conference talk, nobody knows}

\date[CONFname]{YYYY-MM-DD}

% This is only inserted into the PDF information catalog. Can be left out
\keywords{microscopy, Raman, spectroscopy, imaging}

% Use template below if you are the only author
\author[A. Author]{Author Author}
\institute[INST]{
  your department\\
  full and long name of your institute
  }

% Titlegraphic appears in the middle of the titlepage
\pgfdeclareimage[height=2cm]{titlegraphic}{img/logo.png}
\titlegraphic{\pgfuseimage{titlegraphic}}

% Logo; use \insertlogo anywhere you want
%\pgfdeclareimage[height=0.5cm]{university-logo}{img/logo2}
%\logo{\pgfuseimage{university-logo}}


%-------------------------------------------------------------------------------
%                                  Appearance
%-------------------------------------------------------------------------------
% Themes without navi bars: default, boxes, Bergen, Boadilla, Madrid, AnnArbor,
%                           CambridgeUS, Pittsburgh, Rochester
% Themes with tree-like bar: Antibes, JuanLesPins, Montpellier
% Themes with sidebar: Berkeley, PaloAlto, Goettingen, Marburg, Hannover
% Themes with mini frame navi: Berlin, Ilmenau, Dresden, Darmstadt, Frankfurt, Singapore, Szeged
% Themes with sections and subsections tables: Copenhagen, Luebeck, Malmoe, Warsaw, 
\usetheme{CambridgeUS}

% Themes: default, circles, rectangles, rounded, inmargin
\useinnertheme{default}

% Themes: default, infolines, miniframes, smoothbars, sidebar, split, shadow, tree, smoothtree
\useoutertheme{default}

% Special themes: default, structure, sidebartab
% Complete themes: albatross, beetle, crane, dove, fly, seagull, wolverine, beaver
% Inner themes: lily, orchid, rose
% Outer themes: whale, seahorse, dolphin, 
\usecolortheme{crane}

% % % % % Tweaks
%\setbeamercolor{frametitle}{fg=red}
\setbeamertemplate{caption}{\raggedright\insertcaption\par}

% Color of the title
\setbeamercolor{title}{bg=wheat}

% Delete this, if you do not want the table of contents to pop up at
% the beginning of each subsection:
\AtBeginSection[]
{
  \begin{frame}<presentation>{Outline}
    \tableofcontents[currentsection,currentsubsection]
  \end{frame}
}

% Make hidden elements a bit visible
\setbeamercovered{transparent}

% If you wish to uncover everything in a step-wise fashion, uncomment
% the following command:
%\beamerdefaultoverlayspecification{<+->}

% Navigation symbols and frame numbers
\setbeamertemplate{navigation symbols}{
%    \insertslidenavigationsymbol
%    \insertframenavigationsymbol
%    \insertsubsectionnavigationsymbol
%    \insertsectionnavigationsymbol
%    \insertdocnavigationsymbol
%    \insertbackfindforwardnavigationsymbol
%    \hspace{3mm}
     \insertframenumber~/\,\inserttotalframenumber~~\vspace{1mm}}
\setbeamerfont{navigation symbols}{size=\Large}

% Width of blocks
\addtobeamertemplate{block begin}{%
  \setlength{\textwidth}{1.0\textwidth}%
}{}

\addtobeamertemplate{block alerted begin}{%
  \setlength{\textwidth}{1.0\textwidth}%
}{}

\addtobeamertemplate{block example begin}{%
  \setlength{\textwidth}{1.0\textwidth}%
}{}

%-------------------------------------------------------------------------------
%                         Bibliography
%-------------------------------------------------------------------------------
\usepackage[style=authoryear-icomp,
            uniquelist=false,
            uniquename = false,
            dashed=false,
            abbreviate=true,
            dateabbrev=true,
            maxcitenames=1,
            backend=bibtex]{biblatex}
\bibliography{links}


% Select here what you really want to have in the bibliography
\DeclareBibliographyDriver{article}{%
	\usebibmacro{bibindex}%
	\usebibmacro{begentry}%
	\usebibmacro{author}%/translator+others}%
	\setunit{\labelnamepunct}\newblock
	%  \usebibmacro{title}%
	%  \newunit
	%  \printlist{language}%
	\newunit\newblock
	\usebibmacro{byauthor}%
	\newunit\newblock
	\usebibmacro{byeditor+others}%
	\newunit\newblock
	%\printfield{version}%
	\newunit\newblock
	% \usebibmacro{in:}%
	\usebibmacro{journal}%
	\newunit\newblock
	%\printfield{note}%
	\setunit{\bibpagespunct}%
	%\printfield{pages}
	%  \newunit\newblock
	%  \printfield{issn}%
	\newunit\newblock
	%\printfield{doi}%
	%  \newunit\newblock
	%  \usebibmacro{eprint:arxiv}
	%  \newunit\newblock
	%  \usebibmacro{url+urldate}%
	\newunit\newblock
	\printfield{addendum}%
	\newunit\newblock
	\usebibmacro{pageref}%
	\usebibmacro{finentry}}
\DefineBibliographyStrings{ngerman}{
	andothers = {\emph{et\addabbrvspace al\adddot}}            
}

\renewcommand{\cite}{\footfullcite}



